\documentstyle{book}

\title{Lib2geom Manual}

\newcommand{\code}[1]{\textsf{#1}}

\begin{document}

\chapter{Overview}

\section{Introduction}

This manual focuses on the lib2geom computational geometry framework.
The main goal of this framework is the eventual replacement of
Inkscape's halfhazard, shoddy geometry framework. As with any decent
module or lib, it is designed to achieve the necessary functionality
while maintaining a generality which allows useage within other
applications.  The focus on robust, accurate algorithms, as well as
utilization of newer and very useful representations makes the lib
very attractive for many applications.

Along the way we might learn a bit of computational geometry as well.

\section{Design Considerations}

2Geom is written with a functional programming style in mind.
Generally data structures are considered immutable and rather than
assignment we use labeling.  However, C++ can become unwieldly if
this is taken to extreme and so a certain amount of pragmatism is
used in practice.  In particular, usability is not forgotten in the
mires of functional zeal.

The code relies strongly on the type system and uses some of the more
'tricky' elements of C++ to make the code more elegant and 'correct'.
Despite this, the intended use of 2Geom is a serious vector graphics
application. In such domains, performance is still used as a quality
metric, and as such we consider inefficiency to be a bug, and have
traded elegance for efficiency where it matters.

In general the data structures used in 2Geom are relatively 'flat'
and require little from the memory management.  Currently most data
structures are built on standard STL headers\cite{stl}, and new and
delete are used sparingly.  It is intended for 2Geom to be fully
compatible with Boehm garbage collector\cite{boehm} though this has
not yet been tested.

\chapter{Geometric Primitives}

What good is a geometry library without geometric primitives?  By this
I mean the basic stuff that every decent geometry library has -
Points/Vectors, Matrices, etc.

2geom's primitives are descendant from libNR's geometric primitives.
They have been modified quite a bit since that initial import, and
will likely change quite a bit in the future.

\section{Points}

The mathematical concepts of points and vectors are merged into the
2geom \verb|class| called \verb|Point|.  See Appendix A for a further discussion of
this decision.

2geom also breaks with the programming world's more traditional
\verb|struct| with \verb|.x| and \verb|.y| fields.  Rather, each
point has a 2 element array of coordinates.

We found that code using \verb|.x| and \verb|.y| encouraged people
to attempt to inline geometry operations rather than using operators
(perhaps looking for a performance enhancement), and to try and
optimise operations (usually incorrectly).  By using an array, we
encourage people to think about \verb|Point|s as symmetrical objects
and discourage direct use of the components.  We still provide direct
access for the rare occasion that it is needed.  Even in these cases,
the array method prevents bugs by encouraging iteration over the
array rather than explicit element reference.

\section{Transformations}

Affine transformations are either represented with a cannonical 6
element matrix, or special forms selected by the type system.

\subsection{scale}

scale is a pair giving x and y scales.

\subsection{rotate}

rotate is a curl-like vector.

\subsection{translate}

translate is a simple vector.

\subsection{Matrix}

Matrix is a general affine transform.  code is provided for various
decompositions (such as SVD to extract the rotation and scale terms).

\section{Rect}

Rectangles, also known as boxes, have all 4 sides parallel to the axes
and are defined by the least point and the greatest point.

\section{ConvexHull}

two files, convex-hull.h is the inkscape version using rectangles,
convex-cover is partial implementation done with a set of points in
clockwise direction.  Operations:

\begin{description}
\item[empty:] contains no points
\item[singular:] contains exactly one point
\item[linear:] all points are on a line
\item[area:] area of the convex hull
\item[furthest:] furthest point in a direction (lg time)
\item[intersect:] do two convex hulls intersect?
\item[intersection:] find the convex hull intersection
\item[merge:] find the convex hull of a set of convex hulls
\end{description}

\section{Path}

A Path is an ordered set of \code{SubPath}s.  A \code{Path} in
Inkscape is drawn in a single fill and stroke.

\section{SubPath}


We assume that paths are made from $C^0$ continuous segments (called
\code{SubPath::Element} in 2geom).  Because elements are $C^0$ continuous
they can share endpoints.  For storage we have two arrays: handles and
commands.  We assume that all commands define a segment using a number
of \code{Points}, namely on canvas positions, and that affine
transformation of the points will affine transform the segment.  To
reconstruct a path we walk through handles and cmd in parallel,
'eating' enough points for each element adding a handle without
updating cmd would almost certainly make the path garbage.

this has the property of taking only slightly more memory than a
sequence of points in the poly line or poly bezier case, but allows us
lots of flexibility with what sort of elements we can 'understand'.
To walk a SubPath an element a time use the SubPath::iterator.
Inserting directly into \code{SubPath::handles} or \code{SubPath::cmd}
will corrupt the SubPath.

We require each Element provides a specific interface:

\begin{itemize}
\item Affine transforming handles also affine transforms the path.
\item The convex hull of element handles is the convex hull of the elements.
\item The element has a parameterisation $t \in [0,1]$.
\item It is possible split an Element at any $t$.
\item The functions \code{point at} and \code{point and tangent at} exist
\item The Element can be converted to an s-basis.
\item A way to efficiently convert to polylines
\end{itemize}

From these we get a number of useful properties:
\begin{itemize}
\item The convex hull of handles is the convex hull of the path.
\item we can consider a path a single function mapping [0, n] to points.
\end{itemize}

SubPathOps are 'line', 'quadratic bezier', 'cubic bezier', 'c-spline'.

c-spline = funky elliptical representation.

Neither NURBS, nor SVG elliptical arcs can be directly represented in this way.

\section{Location}

\section{Location sequences}

Many algorithms are more efficient on a sorted sequence of locations,
than calling the function repeatedly for each.  So we have algorithms
that take a sequence of locations, assumed in order, and perform an
action on those.  For example, cutting a path at one location is
basically linear in the number of path segments, but cutting a path in
10 locations is still about the same amount of work.  Similarly,
working out the arc length for a location is about the same amount of
work as working out the arc length for 1000 locations on that path.

Many operations are best described as returning an ordered set of
locations - for example, we have a function which returns
intersections between two paths.  rather than return just one
intersection, we might return all intersections, either in order along
the path, or in order of distance along other path.

Think about dashes: a dash is a fixed arclength offset.  So rather
than getting the location for a point at arc length 1, at arc length
2, 3, 4.. to the length of the curve, instead we just ask for all of
these, and the algorithm can chug along the curve outputting the
answer for each.  the reason it is faster is because to work out the
location at arc length say 100, we basically need to work out the
length for many spots up to 100.

Perhaps we then want to split the curve at each of those points.  To
split a segment at a location first requires finding that segment,
then spliting it and finally constructing a new path to output a whole
new path so we can fit the two new segments in.  If we started at the
begining, and split at the first location, that would be n+1 steps - n
segs in the original, plus an extra one.  If we wanted to split at 100
points, it would be n+1 steps for the first, n+2 for the
second... n+100 steps for the last, this would take a total of 100n +
100*101/2 steps!  Whereas, if we split as we went along, it would take
just n+100 steps.

The downside is that I'll probably not provide a separate split
routine that takes a single point, to discourage people from making
exactly that mistake.

\section{S power basis form}

2Geom provides a very powerful algebra for modifying paths.  Although
paths are kept in an extended SVG native form where possible, many
operations require approximation.  To do this we can convert a path
into a sequence of s-power basis polynomials, henceforth referred to
as s-basis, perform the required operations and convert back,
approximating to a requested tolerance as required.

The details of the s-basis form are beyond the scope of this manual,
the interested reader should consult \cite{SanchezReyes1997,SanchezReyes2000,SanchezReyes2001,SanchezReyes2003,SanchezReyes2004} for the details.

The most important properties of the s-basis form are:
\begin{itemize}
\item exact representation of bezier segments
\item fast conversion from all svg elements
\item basic arithmetic - $+$, $-$, $\times$, $\div$
\item algebraic derivative and integral
\item elementary trigonometric functions: $\sqrt{\cdot}$, $\sin(\cdot)$, $\cos(\cdot)$, $\exp(\cdot)$
\item efficient degree elevation and reduction
\item function inversion
\item recovery of exact solutions for many non trivial operations
\item strong convergence guarantees
\item low condition number on bezier conversion.
\item $C^0$ continuity guarantee
\item root finding
\item composition
\end{itemize}

All these operations are fast.  For example, multiplication of two
beziers by converting to s-basis form, multiplying and converting back
takes roughly the same time as performing the bezier multiplication
directly, and further more, subdivision and degree reduction are
straightforward in this form.

\subsection{Surfaces}

2Geom also provides a multivariate form - functions of the form $f(u,v) \rightarrow z$.  These can be used for arbitrary distortion functions (take a path $p(t) \rightarrow (u,v)$ and a pair of surfaces $f(u,v),g(u,v)$ and compose: $q(t) = (f(p(t)), g(p(t)))$.

Subdivision for surfaces are done with either quadtrees or kd-trees.

\section{2D databases}

2Geom provides an implementation of a 2D database using quad trees and
using a list.  Quad trees aren't the best datastructure for queries,
but they usually out perform the linear list.  We provide a
standard interface for object databases with performance guarantees
and provide a set of useful operations Operations:

\begin{description}
\item[Insert:] given a bounding box and a 'reference' insert into the db

\item[Delete:] given a bounding box and a 'reference' delete from the db
both of those should be linear worst case
(quad trees don't guarantee this unless you consider 1024 a constant)

\item[Search:] given a box, find all objects that may interact with this box

\item[Cast:] given a path (including rays) return a list of objects that interact with the path, roughly sorted by path order

\item[Shape query:] given a closed path, find all objects whose bounding boxes intersect path.  (this and cast are nearly the same)

\item[Nearest:] given a point (or maybe box) find the nearest objects, perhaps as a generator to get all objects in order.  To do this, we walk around the quad tree neighbourhood, pushing all the elements into a priority queue, while the queue is empty, move out a bit.  Nearest could be manhattan, max norm or euc?

\item[Binary:] take two dbs, generate all pairs that have intersecting boxes.

\item[Sweep:] traverse the tree in say y order, maintaining a y-range of relevant objects. (to implement sweepline algorithms)

\item[Walk:] traverse the tree in an arbitrary order.

\end{description}


\section{Acknowledgements and history}

2Geom is a group project, having many authors and contributors.  The
original code was sketched out by Nathan Hurst and Peter Moulder for
the Inkscape vector graphics program to provide well typed, correct
and easy to use C++ classes.  Since then many people have refined and
debugged the code.  One of the earliest C++ification projects for
inkscape was replacing NRPoint with NR::Point.

A conspicuous absence was a Path datatype, and indeed Inkscape
developed at least 3 different internal path datatypes, plus several
others in related projects.  Considering the core importance of path
operations in vector graphics this led to much reimplementation of
algorithms, numerous bugs and many round trips converting between
forms.

Many attempts were made to try and develop a single path data
structure, but all were fated to sit in random SCMs scattered across
the web.

Several unrelated projects had copied out various portions of the NR
code from Inkscape and 

In 2006 MenTaLguY and Nathan felt it was time to separate out
the geometry portions of inkscape into a separate library for general
use.  the namespace was changed from NR to Geom and a prototype for
paths was sketched out.  Nathan studied the state of the art for
computational geometry whilst mental focussed on the design of Paths.

Before the remerging of 2Geom with the inkscape svn HEAD it was felt
that a few smaller projects should be ported to use 2Geom.  Michael
Wybrow's libavoid advanced connector routing system was ported first.

--now.

\subsection{People who have contributed to 2Geom}

%alphabetical
Aaron C. Spike,
Fred (livarot),
Javier Sanchez-Reyes,
Jonathon Wright,
Joshua Blocher,
Kim Marriott,
MenTaLguY,
Michael J. Wybrow,
Michael Sloan,
Nathan J. Hurst,
Peter J. R. Moulder

\chapter{Appendix}

\section{A: Geometric Points}
In standard geometry points and vectors are quite distinguished - a  
point is a location, whereas a vector is an unbased direction and
magnitude.  Allowed operations on vectors and points also vary:

P  - P = V
P +- V = P
V +- V = V
V */ S = V

, where P represents points, V represents vectors, and S scalars.

Ideally we would render these restrictions in code, as they would
reinforce algorithm correctness.  This is because as far as arithmetic
operations go, the above are all that you sanely require, unless you
are prematurely optimizing.

\end{document}
